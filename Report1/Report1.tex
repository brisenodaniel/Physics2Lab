\documentclass[12pt]{amsart}
\usepackage[english]{babel}
\usepackage{graphicx}
\usepackage{float}
\usepackage{mathtools}
\usepackage{amsfonts}
\usepackage{amssymb}
\usepackage{siunitx}
\usepackage{amsthm}
\usepackage{enumitem}
\usepackage{stmaryrd}
\usepackage{multirow}
\usepackage[backend=bibtex,style=numeric]{biblatex}
\bibliography{Biblio}
\usepackage[a4paper, total={6in, 10in}]{geometry}
\graphicspath{{./}{}}% You can add the path for the images in the empty brackets 
\title{Study of Electrostatics: Generation and Measurement of Charge}
\author{Josh Goldfaden, Daniel Briseno}
\date{}
\newdimen\graph
\graph=4.2in
\newdimen\medgraph
\medgraph = 5.3in
\newdimen\smallgraph
\smallgraph = 3in
\newdimen\tinygraph
\tinygraph = 1.5in
\renewcommand{\arraystretch}{1.5}
\begin{document}
\maketitle
\section{abstract}
In this set of experiments, the flow of electric charge was studied through initially conducting demonstrations with charge generator machines. In particular, a Van de Graaff Generator was used to illuminate a neon light bulb without making direct contact. This observation demonstrated that there is a flow of electric charge between the generator and the bulb. When bringing a neutrally-charged conductive strip or a ball into close proximity with the generator, the two objects attracted. However, once the two objects made contact, they repelled one and another. Once the objects made contact, charge was transferred from the generator to the object, thus the like charges repelled. Further, when an Electric Plume was placed on the generator, the ribbons threaded through the top of the Electric Plume slowly began to rise and repel one another and the plume as all acquired charge from the generator. The electrostatic force emitted by the generator also induced the movement of the Electric Whirl Spin when brought into close proximity. Another electrostatic machine studied was the Wimshurst Influence Machine. When this machine was cranked, an electric charge was generated and was visible as a spark between the two terminals. The presence of an electric charge was also observed through the use of an electroscope; in particular, the two strips of metal concealed within the apparatus repelled one another when a charged object was brought into contact with a wire suspending each strip. Aside from these apparatuses, various objects (e.g. pieces of metal, wool/fur, glass, wood, etc.) were charged by being rubbed together, and these charges were measured using a Vernier Charge Sensor integrated with a Faraday Pail. From this, it was determined that ----. Further, this apparatus was used to measure the quantity of charge that resides in an individual’s body. It was observed that ----- when an individual scuffed their shoes on a piece of carpet.
 
\section{Introduction}
Electrostatics is a fundamental branch of physics that examines stationary electric charges that generate an electrostatic field. In this particular investigation, an array of apparatuses are utilized to observe the generation of an electrostatic field. One paramount exemplar of such a device is the Van de Graaff generator rendered in Figure 1.\\
\begin{figure}[h]
	\includegraphics[width=\smallgraph,scale=0.01]{Van.png}
	%h (here) - same location
	%t (top) - top of page
	%b (bottom) - bottom of page
	%p (page) - on an extra page5
	%! (override) - will force the specified location
	\caption{The basic structure of a Van de Graaff generator. Photo Credit  \cite{Vander}}
	\label{Van}
\end{figure}

\indent Within the generator, a moving belt is wound around a plastic pulley. When the motor is in use, it rotates the pulley, which consequently rubs against the belt and induces its motion. The contact between the belt and the pulley results in the generation of static electricity. Particularly, negative charges from the belt is transferred to the plastic pulley.  This acquisition of charge by the pulley is the triboelectric effect in action. That is, two dissimilar objects, the belt and the pulley, come into contact with one another, and the pulley acquires a charge when separated from certain regions of the belt that it was previously in contact with \cite{Vander:2}. As the pulley becomes more negatively charged, it induces a positive charge within the metal brush at the base of the generator through frictional contact. Specifically, the negatively charged pulley is in close proximity to the neutrally charged metal brush. The charged pulley drives the redistribution of electric charge in the metal brush such that the brush acquires a positive electric charge. In essence, an electrostatic field is generated between the plastic pulley and the metal brush, thus causing the air surrounding the brush within the base of the generator to become ionized. The positive charges of the air repel the positively charged brush and accumulate on the surface of the moving belt. These positive charges are subsequently moved by the belt to the dome of the generator. The positive charges on the belt are transferred to the dome via air ionization and the metal brush suspended within the dome. This results in the collection of  positive charge on the surface of the dome, thus increasing its electric potential. Conversely, the Van de Graaff generator can also become negatively charged, depending on the material of the belt and the pulley.\\

\indent In an instance of a negatively charged Van de Graaff generator, the electrons spread across the dome repel one another. When a neon light bulb is brought in proximity to the dome, the electrons are attracted to the anode (the positively charged electrode) of the light bulb\cite{scienceworld}. This movement of electrons results in an electric current between the dome of the generator and the bulb, thus inducing the bulb’s illumination. Additionally, the individual holding the bulb also assists electron flow. Another accessory to the Van de Graaff generator that can be used to observe electrostatics is an Electric Plume, which is mounted on a generator in Figure 2.\newpage
\begin{figure}[h]
	\includegraphics[width=\smallgraph,scale=0.01]{Plume.png}
	%h (here) - same location
	%t (top) - top of page
	%b (bottom) - bottom of page
	%p (page) - on an extra page5
	%! (override) - will force the specified location
	\caption{ An Electric Plume mounted on the dome of a Van de Graaff generator. Photo Credit \cite{Flinn}}
	\label{Plume}
\end{figure}

\indent Electrons are transferred from a negatively charged dome onto the neutrally charged ribbons of the plume. As the ribbons acquire like charges, they begin to repel one another and appear to be suspended in the air as visualized in Figure 2. Electrons that accumulate on the dome of the Van de Graaff generator are also responsible for driving the movement of the Electric Whirl Spin, which is depicted in Figure 3.\\

\begin{figure}[h]
	\includegraphics[width=\smallgraph,scale=0.01]{Whirl.png}
	%h (here) - same location
	%t (top) - top of page
	%b (bottom) - bottom of page
	%p (page) - on an extra page5
	%! (override) - will force the specified location
	\caption{One variant of the Electric Whirl Spin. Photo Credit \cite{greenslade}}
	\label{Whirl}
\end{figure}

\indent When brought in proximity to a negatively charged Van de Graaff generator dome, electrons escape from the dome and accumulate onto the pointed ends of the Whirl Spin’s arms. This causes the air around the pointed ends to become ionized and is thus repulsed by the arms. This repulsion by the arms induces their rotation about the pivot they are attached to. Thus, the Electric Whirl Spin begins to twirl \cite{davis_2012}. \\

\indent When a neutrally charged object, such as a pith ball, is brought in proximity to the dome of  Van de Graaff generator, the two objects will be attracted as observable in Figure 4.\\

\begin{figure}[h]
	\includegraphics[width=\smallgraph,scale=0.01]{Pith.png}
	%h (here) - same location
	%t (top) - top of page
	%b (bottom) - bottom of page
	%p (page) - on an extra page5
	%! (override) - will force the specified location
	\caption{A neutrally charged pith ball moving towards a charged Van de Graaff generator. Photo Credit \cite{dickie}}
	\label{Pith}
\end{figure} 

\indent This attraction between these two objects is the result of electrostatic induction in that equal amounts of positive and negative charge on the ball are redistributed. For instance, a positively charged Van de Graaff generator will attract the negative charge on the pith ball, thus causing this charge to accumulate on the side of the ball facing the generator. As for the positive charge on the pith ball, this is repelled by the like charged generator thus it accumulates on the side of the ball that is furthest from the generator. However, once the pith ball makes physical contact with the generator, positive charge from the generator is transferred to the ball, thus causing the two objects to repel. 

Another electrostatic generator utilized in this lab is the Wimshurst influence machine, which is depicted in Figure 5. \\
\begin{figure}[h]
	\includegraphics[width=\smallgraph,scale=0.01]{Wim.png}
	%h (here) - same location
	%t (top) - top of page
	%b (bottom) - bottom of page
	%p (page) - on an extra page5
	%! (override) - will force the specified location
	\caption{Wimshurst influence machine components from the front. Photo Credit \cite{Wim}}
	\label{Wim}
\end{figure} 

\indent Unlike the mechanism of the Van de Graaff generator, the Wimshurst influence machine does not rely on friction. Rather, similar to the interaction between a charged Van de Graaff generator and a pith ball depicted in Figure 4, the Wimshurst influence machine generates electrostatic charge through induction. Specifically, the triboelectric effect must be used to induce a charge on the sectors of either disk. Once there is an imbalance in the charge between each disk, the disks can be rotated through manually operating the crank. This crank drives the motion of a pulley system that induces the rotation of the disks. One of the pulley’s belts is twisted causing one disk to spin in the opposite direction of the other disk. Each disk’s sectors have either an excess of positive or negative charge. As the disks rotate, induction occurs between the sectors on each disk. For instance, a negatively charged sector induces the adjacent sector on the neighboring disk. The side of the adjacent sector that is closest to the negatively charged sector accumulates positive charge. The negative charge on this sector is simultaneously repelled and travels along a diagonally positioned neutralizer that is in contact with another sector on the same disk. This repelled negative charge travels to the sector, which causes induction on the neighboring disk. Overall, repeated induction occurs between the sectors of each disk. The positive and negative charge that accumulates on the individual sectors induce the two collectors that are suspended around the disks at opposite ends. The combs of each collector either accept electrons from negatively charged sectors or donate electrons to positively charged sectors. This exchange of electrons creates an electric field between the collectors and the sectors of the disk. The positive and negative charge that build up in the two respective collectors are distributed to two different Leyden jars. These Leyden jars, which serve as capacitors, are composed of an inner metal cylinder, an insulator, and an outer metal cylinder. The two metal cylinders acquire opposite charges, but no neutralization can occur as the insulator between them prevents the exchange of any charges. Similarly, the spark gap between the two conductors mounted on the Leyden jars also serve as a capacitor where air is an insulator. It is noteworthy that increasing the humidity of the air may result in the leakage of electric charge as water is a good conductor, thus limiting the amount of static electricity generated. An electric field is generated in this gap, and the flow of charged particles creates an electric spark. This results in the neutralization of each Leyden jar, and this cycle repeats itself as the crank is continually spun \cite{rim}. \\

\indent A class of  apparatuses that are utilized to determine the presence of electric charge are known as electroscopes. The variant used in this lab is rendered in Figure 6\\
\begin{figure}[h]
	\includegraphics[width=\smallgraph,scale=0.01]{leaf.png}
	%h (here) - same location
	%t (top) - top of page
	%b (bottom) - bottom of page
	%p (page) - on an extra page5
	%! (override) - will force the specified location
	\caption{An aluminum foil leaf electroscope. Photo Credit \cite{Ele}}
	\label{leaf}
\end{figure} 

\indent An object with a suspected static electric charge, such as a balloon that has been rubbed against an individual’s hair, is brought near the copper wire of the electroscope. Electrons from the charged balloon are transferred to the copper wire and flow down the wire. At the end of the wire within the jar, the electrons are finally transferred to the leaves of aluminum foil which consequently gain negative charges. These leaves of negatively charged aluminum foil repel one another, thus indicating the presence of static electric charge on the balloon \cite{Champ}. \\

\indent Interestingly, an individual’s body can also serve as an electrostatic conductor and carries charge. The amount of charge on a human body as well as other suspectedly charged objects can be quantified using a Faraday Pail, a schematic of which is depicted in Figure 7. \\
\begin{figure}[h]
	\includegraphics[width=\tinygraph,scale=0.01]{Farday.png}
	%h (here) - same location
	%t (top) - top of page
	%b (bottom) - bottom of page
	%p (page) - on an extra page5
	%! (override) - will force the specified location
	\caption{A diagram of a Faraday Pail integrated with an electrometer.}
	\label{Farday}
\end{figure} 

\indent This apparatus demonstrates that an electric charge encompassed by a conducting shell produces an equal charge on the shell, and the charge of an electrically conducting object resides entirely on its surface. Further, this device also provides a quantification of electrostatic charge. An individual’s body can be grounded and integrated with a ground plane upon which a Faraday Pail sits. When an individual creates a charge by scuffing their shoes on a piece of carpet, this charge can be measured by the device.\\

\indent In this study of electrostatics, we observed the movement of charge between objects by means of friction and induction. In particular, we studied the generation of static electricity using a Van de Graaff generator, a Wimshurst influence machine, and an electroscope. Utilizing a grounded Faraday Pail integrated with a Vernier Charge Sensor, we quantified the charge of various objects as well as our own bodies, which were grounded to the metal plane upon which the Faraday Pail sat. These experiments have provided us an opportunity to not only have a less abstract perception of static electricity, but we have a greater understanding and appreciation for the various physical phenomena that occurs in the world around us.

\section{Electrostatic Experiments}
\indent In this set of experiments, we wanted to further study and quantify the electrostatic charges that exist upon various objects, including our bodies. As humidity influences the flow of charge, we measured the relative humidity of the lab at 51.66\%. In order to measure electrostatic charge, we placed the cage encompassing the Faraday Pail on top of a metal grounding plane, and measured the charge of the pail (and thus indirectly measured the charge of anything placed in the pail). See figure 8 for more details. \\


\begin{figure}[h]
	\includegraphics[width=\medgraph,scale=0.01]{FardaySetup.png}
	%h (here) - same location
	%t (top) - top of page
	%b (bottom) - bottom of page
	%p (page) - on an extra page5
	%! (override) - will force the specified location
	\caption{Faraday Pail Setup. The charge sensor is set to $\pm$10 V}
	\label{FardaySetup}
\end{figure} 

\subsection{Measuring Charge on Our Own Bodies}
\indent We first attempted to measure the charge of our own hands by inserting our finger into the Faraday Pail. Our procedures for this part of the experiment can be summarized as follows:
\begin{enumerate}
	\item Scuff shoes on laboratory floor.
	\item Insert a finger into the pail without touching the pail or cage and record charge sensor readings.
	\item Remove finger from pail and record charge sensor readings
	\item Touch the metal grounding plate with your finger, then repeat steps (2) and (3).
	\item Ground the alligator clip on the grounding strap to the grounding plate and attach grounding strap to wrist.
	\item Repeat steps (1)-(4)\\
\end{enumerate}

\indent Unsurprisingly, none of the measurements taken while wearing the grounding straps showed a significant change in charge when we inserted a finger in the Faraday Pail. This can be attributed to the fact that the grounding strap ensures that our bodies have the same electric charge as the grounding plate, and thus should cause no change in the charge of a neutral pail.\\
\indent However, the measurements taken without wearing the grounding strap showed no significant changes in the charge of the pail as well. While this somewhat surprising considering that we were not grounded when making these measurements, it may be explained by 2 factors: 1 -- the relatively high humidity in the lab may have been enough to discharge our bodies to the surrounding air, making us more or less continuously grounded; 2 -- the sensor gave readings varying from 0-30nC even after grounding the pail and keeping the pail empty (such that the reading should have been 0nC), showing that the sensor may not have been giving accurate data and thus masked a significant change in charge. 

\subsection{Separation of Charge}
\indent After measuring the electrostatic charge on our own bodies, we measured the charge obtained by different material when rubbed against a dissimilar object. We rubbed a wooden rod, a PVC rod, and a nylon rod against Vinyl, Felt, and a cotton washcloth. We then measured the charge obtained by the rods after being rubbed with the different materials:
\begin{table}[H]
	\begin{tabular}{ |c|c||c|c|}
		\hline
	
		\hline
		Rod Material & Material rubbed on rod& Type of charge (+,0,-)&Amount of Charge (nC)\\
		\hline
		\multirow{3}{*}{Wooden Rod} &Vinyl&0&0\\
		 &Felt&0&0\\
		&Washcloth&0&0\\
		\hline	
		\multirow{3}{*}{PVC Rod}&Vinyl&-&24.80\\
		&Felt&-&33.00\\
		&Washcloth&-&14.40\\
		\hline
		\multirow{3}{*}{Nylon Rod} &Vinyl&+&5.00\\
		&Felt&-&1.30\\
		&Washcloth&0&0\\
		\hline	
	\end{tabular}
\caption{Charge obtained by various rods after rubbing on different materials}
\end{table}

\indent As seen in Table 1, the wooden rod, even when rubbed with a sheet of vinyl, felt, and a washcloth, failed to obtain a charge. One may speculate that the wooden rod's inability to hold an electrostatic charge is due to wood being a poor conductor. Atomically speaking, the electrons of the atoms that compose the wooden rod are unable to move freely such that they can generate an electrical charge. However, the PVC rod and the Nylon rod were both able to hold a charge, and both are also poor conductors.\\
\indent Unlike the nylon and PVC rods, the wooden rod was quite porous. The presence of air within the rod may have allowed the wooden rod to discharge into the environment, as well as limiting the amount of surface area in contact with the rubbing fabrics. Therefore, we believe that the wood's inability to hold an electrostatic charge was due to the wood being porous rather than a bad conductor.\\

\indent With respect to the charges measured that correspond to the PVC rod being rubbed with the materials listed previously, it is observable that the PVC rod acquired a negative charge after being rubbed by all 3 types of fabrics. However, the amount of charge deposited on the PVC rod by the different fabrics was different, with the felt depositing the most and the washcloth the least.
 
\indent Regarding the charges measured that correspond to the nylon rod being rubbed with these same materials, it was found that when rubbed with vinyl, the rod acquired a small positive charge. When rubbed with felt, the nylon rod acquired a small negative charge. When rubbed with a washcloth, the nylon rod remained neutral in charge.\\

\indent Recall from the introduction that the triboelectric effect (the separation of charge caused by rubbing two dissimilar objects together) is a result of small chemical interactions between the dissimilar objects being rubbed. More specifically, when two dissimilar objects are brought into contact there are some chemical bonds formed between negatively charged regions of one object and positively charged regions of the other. Because of these chemical bonds, some negative charge is ''pulled" from one object to the other, leaving the ''pulling" object negatively charged and the other positively charged. Since chemical interactions are a result of \textit{both} object's atomic structure, it makes sense that the PVC, wood and nylon rods all reacted differently to the different rubbing materials and that in some cases, each rod acquired different charges from the three materials.\\

\indent During this experiment we also attempted to measure the charge separation created by: taking two tapes, taping one to a table and taping the other on top of the first, quickly taking both tapes off the table, then quickly separating the two tapes. Indeed, the tapes acted as if at least one of them was charged since there was an observable attractive force between them. However, we were unable to measure any charge using the Faraday Pail apparatus. Given the sensor inaccuracy mentioned previously, it is unsurprising that the charge of both the T Strip and the B Strip was unable to be determined.\\

\subsection{Movement of Charge}

\indent After observing charge separation caused by rubbing dissimilar objects, we investigated other ways in which an electric charge might be manipulated\footnote{Note that before beginning each experiment, we used the Faraday Pail apparatus to measure the charge of each separator and ensure that they were neutral. However, we believe that the sensor gave inaccurate data and the separators may have been slightly charged before the first experiment. }.In order to do this, we rubbed two charge separators together and then measured their charge using the Faraday Pail apparatus. The results of these measurements are shown in Figure 9.\\


\indent Figure 9 shows that the white charge separator acquired a positive charge while the gray separator acquired a negative charge. the white charge sensor has a slightly greater positive charge than the gray charge sensor has a negative charge. Thus, when both immersed in the pail simultaneously, it is expected that a slight positive charge is measured. Since rubbing the two charge separators could not have created positive charge, but rather separate the charge present in the separators, we believe that the charge separators may not have been neutral when we began this experiment, and thus had the net positive charge seen when both separators are measured.\\



\begin{figure}[h]
	\includegraphics[width=\medgraph,scale=0.01]{SeparatorsBor.png}
	%h (here) - same location
	%t (top) - top of page
	%b (bottom) - bottom of page
	%p (page) - on an extra page5
	%! (override) - will force the specified location
	\caption{A charge (nC) versus time (s) plot generated when immersing and removing the charged gray and white charge separators in the grounded Faraday Pail}
	\label{BOr}
\end{figure} 
\indent In the next phase of the experiment we charged the Faraday Pail by contact. We did so by first charging the charge separators, then touching the pail with the white charge separator, removing that separator, then touching the pail with the gray charge separator. The results of this can be seen in Figure 10.\\

 
\begin{figure}[h]
	\includegraphics[width=\medgraph,scale=0.01]{SeparatorsContact.png}
	%h (here) - same location
	%t (top) - top of page
	%b (bottom) - bottom of page
	%p (page) - on an extra page5
	%! (override) - will force the specified location
	\caption{ A charge (nC) versus time (s) plot generated when touching a Grounded Faraday Pail with the charged gray and white charge separators. Note that when both charge separators made contact with the pail, they did not come into contact with each other.}
	\label{Contact}
\end{figure} 
\newpage
\indent The graph shows a charge distribution that is in line with our understanding of electrostatics, and can be interpreted as follows:
\begin{enumerate}
	\item The Faraday Pail apparatus begins with a neutral charge.
	\item The white charge separator touches the pail and discharges positive charge onto the pail\footnote{For the sake of simplicity we will say that the positive charge is deposited onto the pail, although the pail is discharging electrons onto the charge separator which has a surplus of positive ions.}.
	\item The white separator is removed, and the pail is left with a positive charge.
	\item The gray separator touches the pail and discharges negative charge onto the pail.
	\item The gray separator is removed, and the pail is left with a negative charge.  
\end{enumerate}

\indent It may seem strange that the pail is left with a surplus negative charge when the entire system began with neutral charge (including the charge separators). However, recall that each charge separator has exactly enough charge to cancel the charge on the other separator. Also note that the pail begins with nearly neutral charge. Thus, when the white charge separator contacts the pail, the pail is left with less positive charge than the white charge separator had prior to contact\footnote{Since the charge will be equally distributed across the separator and the pail.}. As a result, when the negatively charged gray separator contacts the pail, the pail does not have enough positive charge to render it and the gray separator neutral, and it is left with a negative charge.\\

\indent Finally, we charged the Faraday Pail via induction. We did so by first confirming that the charge separator had an electric charge by measuring it using the Faraday Pail. Then, we again inserted the charge separator into the pail without allowing the separator to contact the pail. With the charge separator still in the pail, we briefly grounded the pail, then after disconnecting it from ground, we removed the charge separator. The results shown in Figure 11.
\begin{figure}[h]
	\includegraphics[width=\medgraph,scale=0.01]{SeparatorInduction.png}
	%h (here) - same location
	%t (top) - top of page
	%b (bottom) - bottom of page
	%p (page) - on an extra page5
	%! (override) - will force the specified location
	\caption{A charge (nC) versus time (s) plot generated when immersing and removing a charged white charge separator in a Faraday Pail and subsequently touching the pail with a grounding wire. Note that the white charge separator did not make contact with the pail.}
	\label{Induction}
\end{figure} 

\indent From Figure 11, we can see that this process left the pail with a negative charge. This can be explained as follows:\\
\begin{enumerate}
	\item The charge separator is inserted into the pail, making the entire system positively charged despite the fact that the pail itself remains neutral.
	\item The pail is briefly grounded. The positive charge from the charge separator attracts negative charge from ground and negative charge flows onto the pail. The charge of the entire system approaches neutral
	\item The grounding wire is removed. The system remains near neutral but the electrons in the pail can no longer leave the pail.
	\item The charge separator is removed. Without the positive charge from the charge separator, the electrons which were attracted to the charge separator now give the pail a negative charge.
\end{enumerate}

\indent Note that although the pail apparatus appears to be neutral at the end of the experiment, the apparatus began at +0.5nC despite being grounded before the experiment began. Thus the sensor was giving charge readings more positive than the actual apparatus had, and it is likely that the final reading would have been negative with a more accurate sensor.

\subsection{Non-Triboelectric Experiments}
\indent Up until now we had only been studying electric charge caused by friction. In this series of experiments we used a high voltage electrostatics kit to charge two metal conducting spheres and conducted experiments using the spheres.\\
\indent In the first experiment we merely verified that we could, in fact, charge the spheres by touching them with a voltage terminal. We touched a sphere with a voltage terminal attached to a 750, 1500, 3000, and 6000 volt source. The results are shown in Table 2.\\
\begin{table}[H]
	\begin{tabular}{ |c|c|}
		\hline
		Terminal Voltage(V) & Charge(nC)\\
		\hline
		750&0.50\\
		1500&0.90\\
		3000&2.00\\
		6000&3.30\\
		\hline	
	\end{tabular}
	\caption{Charge of Proof Plane Measured by Faraday Pail Charge Sensor Apparatus}
\end{table}

\indent As can be seen in the table, we were in fact able to produce an electrostatic charge on the spheres. This can be explained by the fact that the spheres were neutral at the beginning of the experiment. Therefore, touching them with a negatively charged voltage terminal would deposit a negative charge on the sphere, with higher voltages depositing more charge.\\

\indent Next we observed the charge distribution on a single sphere. We did so by touching a sphere with a 3000V Voltage terminal, then touching various parts of the sphere with a proofing plane and measuring the charge left on the proofing plane. The results are shown in Table 3.

\begin{table}[H]
	\begin{tabular}{ |c|c|}
		\hline
		Location of Sample & Charge(nC)\\
		\hline
		Top of Sphere&2.70\\
		Bottom of Sphere&2.50\\
		Side of Sphere&2.70\\
		\hline	
	\end{tabular}
	\caption{Charge obtained on proof plain by touching charged sphere in various locations.}
\end{table}

\indent In contacting the sphere at different points with a charged proof plane, the charge of the metal sphere remains the same, as rendered in Table 7. In trial 2, it is observable that the charge of the sphere is approximately 0.20 nC less than the charges recorded during trial 1 and trial 3. This minor discrepancy is likely a result of the charge sensor not being calibrated to a 0 nC charge prior to taking the trial 2 measurement. Ultimately, these consistent results indicate that the charge acquired from the proof plane by the sphere is evenly distributed along its surface, regardless of where on the sphere the charge is transferred. \\

\indent In  the next phase of the experiment we sought to observe the distribution of charge on spheres $S_1$ and $S_2$. We did so by first touching $S_1$ with a 3000V terminal, then measuring the charge on the sphere using the Proof plane and the Faraday Pail apparatus. We then brought $S_1$ briefly into contact with $S_2$ and measured the resulting charge on $S_1$ and $S_2$. The results of this are shown in Table 4.\\

\begin{table}[H]
	\begin{tabular}{ |c|c|}
		\hline
		 Condition& Charge(nC)\\
		\hline
		$S_1$ before touching $S_2$&1.70\\
		$S_1$ after touching $S_2$&1.10\\
		$S_2$ after touching $S_1$&0.70\\
		\hline	
	\end{tabular}
	\caption{Charges on Sphere 1 ($S_1$) and Sphere 2 ($S_2$)}
\end{table}

\indent As Table 4 shows, the magnitude of the charge on $S_1$ dropped after being brought into contact with $S_2$. This can be explained by some of the surplus negative charge on $S_1$ discharging onto $S_2$. After being brought into contact, $S_1$ and $S_2$ both had negative charges.\footnote{We would have expected the charges on both $S_1$ and $S_2$ to have been the same after being brought into contact. We did not see this, but we again suspect sensor error since our Faraday Pail apparatus had been giving varying readings even when grounded at that time.}\\

\indent Finally we concluded our experiments by charging $S_2$ by induction. We did so by touching $S_1$ with a 3000V terminal, thus giving it a negative charge. We then brought $S_2$ close to $S_1$ but did not allow the spheres to touch. Without moving $S_2$ away from $S_1$, we touched $S_2$ with our finger, then removed our finger before moving $S_2$ away from $S_1$. Finally, we measured the charge on both $S_1$ and $S_2$. The results are shown in Table 5.

\begin{table}[H]
	\begin{tabular}{ |c|c|}
		\hline
		Sphere & Charge(nC)\\
		\hline
		$S_1$&-0.20\\
		$S_2$&1.20\\
		\hline	
	\end{tabular}
	\caption{Charges of spheres after charging $S_2$ by induction}
\end{table}

\indent The table clearly shows that $S_2$ acquired a positive charge after coming into close proximity with $S_1$\footnote{Here we see that the charge acquired by $S_1$ is surprisingly large, while the charge remaining on $S_1$ is surprisingly small. We again suspect sensor error and believe that the charges were more negative than recorded.}. This can be explained by $S_2$ becoming polarized as it is brought close to $S_1$. After being brought close to $S_1$, the side facing $S_1$ became positively charged and the side facing away from $S_1$ became negatively charged. By touching $S_2$ on the side away from $S_1$, we allowed $S_2$ to discharge onto our finger. By removing our finger, we did not allow electrons to re-enter $S_2$ when the polarizing $S_1$ was taken away, thus leaving $S_2$ positively charged.

\section{Conclusion}
\indent In the first portion of this report, we observed the Triboelectric effect. Although we were unable to generate a measurable charge by scuffing our shoes on the laboratory floor, we did observe significant charges being deposited on different rods after being rubbed by vinyl, felt and a cotton washcloth. In particular, we observed that the PVC rod was readily charged, the Nylon rod resisted being charged more than the PVC, and the Wooden rod failed to acquire a charge at all. These results are suggestive that the composition of a material is responsible for its acquisition of charge from the triboelectric effect. We later used charge separators to study the movement of charge across materials. We first found that charge acquired from one charge separator came from the other charge separator, as the two separators had opposite charges of similar magnitude after being rubbed together. We then observed how the two charge separators could charge a different object, both by induction and contact. Specifically, we saw how although the system consisting of the charge separators and the Faraday Pail was neutral as a whole, we could manipulate the distribution of charge to leave the Faraday Pail with a non-neutral charge. Finally, we used a high-voltage source to conduct experiments with two metal conducting spheres, $S_1$ and $S_2$, and observed how charge was evenly distributed across a single sphere, but the distribution of charge across both spheres could be manipulated much in the same way as in the charge separators.
\newpage

\printbibliography

	
\end{document}